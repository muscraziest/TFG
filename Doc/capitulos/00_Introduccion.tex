\chapter*{Introducción}
\addcontentsline{toc}{chapter}{Introducción}

La música es un arte que ha acompañado al ser humano desde casi sus inicios. Ésta se ha ido desarrollando a lo largo de los siglos, adaptándose a la época y sus avances, creando nuevos instrumentos y nuevos estilos. Con la evolución de los distintos estilos musicales se fue desarrollando de igual manera la notación musical, desde las primeras notaciones creadas por la cultura griega, a base de letras del alfabeto y algunos símbolos como puntos, hasta la notación abstracta utilizada hoy en día. 

En la actualidad, gracias a las nuevas tecnologías, disponemos de editores de partituras que realizan la misma función que los editores de texto que encontramos en cualquier ordenador: poder crear y editar una partitura. Sin embargo, una de las funcionalidades que ofrecen los editores de texto y también los móviles son los correctores ortográficos y/o gramaticales, los cuáles en los dispositivos como smarphones y tablets son ampliamente utilizados. 

Algunos editores de partituras, como \textit{Sibelius}, en sus últimas versiones sí añaden parte de esta funcionalidad, pudiendo buscar algunos tipos de errores en partituras, tales como quintas u octavas directas. Sin embargo, las reglas de armonía clásica en las que se basan la mayoría de composiciones, salvo las basadas en nuevas corrientes musicales como música contemporánea o experimental, exponen una gran cantidad de restricciones sobre cómo debe componerse una pieza musical de estilo clásico para que sea correcta y coherente.  Estas reglas tratan desde la correcta disposición de las notas de un acorde hasta la conducción de melodías y la búsqueda de sonoridades consonantes, evitando las disonancias o utilizándolas de maneras determinadas para conseguir efectos sonoros específicos. 

El interés de este proyecto radica pues en la falta de una herramienta de este tipo que permita a compositores y estudiantes de composición y armonía poder corregir errores en sus composiciones o ejercicios de forma automática, con el objetivo de mejorarlas.

Dado que la música abarca una gran variedad de estilos y formas musicales, he decidido para este proyecto centrarme en una de las más extendidas y utilizadas dado su interés para la enseñanza de este campo: los corales o partituras de coro. Esta forma musical proporciona el entorno perfecto para poder estudiar las reglas armónicas, melódicas y de contrapunto al tener cuatro melodías cantadas por cuatro voces independientes que se van entrelazando a lo largo de la obra. 

La realización de este proyecto se centra en la resolución de un problema muy concreto con un entorno muy específico. Además, aunque hay reglas de armonía que son completamente indiscutibles, existen gran cantidad de excepciones a éstas y cierta subjetividad; algo puede no estar necesariamente mal pero no llegar a ser del todo correcto. Un caso sería la conducción melódica de las voces, la cual puede hacerse de diversas maneras según lo que quiera transmitir el compositor en un momento determinado.

Por estos motivos, mi propuesta para poder implementar esta funcionalidad se basa en un sistema experto, el cual a partir de una base de conocimiento de teoría musical y reglas armónicas pueda analizar las partituras de igual manera que un experto en esta materia, teniendo en cuenta la posible subjetividad del análisis.

En los siguientes capítulos se explicará detalladamente el proceso de desarrollo de este sistema, así como los resultados obtenidos tras su implementación. 