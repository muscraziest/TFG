\chapter{Pruebas y validación}

Con toda la implementación del sistema realizada, pasamos a la fase de verificación y validación del sistema. 

\section{Verificación}

Como describimos en el capítulo anterior, la verificación del sistema experto consistió en comprobar que accedía y leía bien el contenido de los ficheros y añadía los hechos a la memoria de trabajo de manera correcta. 

Como no podíamos comprobarlo visualmente como hicimos con el prototipo, se hicieron pruebas en las que el sistema experto, además de los errores, mostraba los hechos asertados, hechos calculados y hechos leídos. 

Una vez se comprobó que el sistema experto se ejecutaba correctamente, pasamos a comprobar las funcionalidades de la interfaz.

Primero se comprobó que las funciones de extracción de información del archivo XML obtenían y guardaban correctamente los datos. Para ello, se ejecutaron todas las funciones con distintas partituras y se revisaron los ficheros resultantes para ver si coincidían con el formato escogido y no contenían fallos.

También se verificó el funcionamiento del envío del formulario para que no se produjesen las siguientes situaciones:

\begin{itemize}

	\item No se puede enviar un formulario en blanco.
	\item No se puede enviar un archivo sin seleccionar ninguna opción.
	\item No se puede enviar la lista de opciones sin adjuntar un archivo.
	\item El archivo a adjuntar debe ser de formato XML.

\end{itemize}

Por último, se examinó que los resultados se mostrasen de manera legible y ordenados en sus pestañas correspondientes. Las pestañas de los módulos deben mostrarse si se ha seleccionado dicho módulo y si se han obtenido errores. En caso contrario, no deberían mostrarse.

\section{Validación}

La validación del sistema experto fue llevada a cabo por Clara Luz Fernández Vecino, profesora de armonía del Conservatorio Ángel Barrios de Granada.

Para ello, el sistema analizó ejercicios realizados por los alumnos del conservatorio:

\begin{itemize}

	\item Alrededor de 100 ejercicios de armonía. Con estos ejercicios se validaron los módulos de errores provocados por los movimientos y algunos errores armónicos. 

	\item 20 ejercicios de composición de corales. Con estas composiciones se validaron todos los módulos, especialmente el módulo de análisis melódico.

\end{itemize}

Tras ejecutar el sistema, Clara Luz revisó los resultados obtenidos por el sistema comparándolos con las correcciones que ella misma había llevado a cabo sobre todos los ejercicios.