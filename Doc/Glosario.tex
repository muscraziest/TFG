\chapter*{Glosario}
\addcontentsline{toc}{chapter}{Glosario}


\begin{itemize}[label={}, leftmargin=*]

	\item \textbf{Acorde}: conjunto de tres o más notas diferentes que suenan simultáneamente y que constituyen una unidad armónica. Las notas que forman los acordes de triada o tres notas se denominan: \textit{fundamental} - nota sobre la que se construye el acorde -,  \textit{tercera} - nota que forma un intervalo de tercera con la fundamental - y \textit{quinta} - nota que forma un intervalo de quinta con la fudamental. En el caso de acordes de cuatriada o cuatro notas, además de las notas que se mencionan, aparece la \textit{séptima} - nota que forma un intervalo de séptima con la fundamental.

	\begin{itemize}

			\item \textbf{Acorde invertido}: se dice que un acorde está \textit{invertido} cuando la nota más grave no es la fundamental, sino alguna de las otras que forma el acorde. Si se forma sobre la tercera, el acorde está en \textit{primera inversión}; si se construye sobre la quinta, está en \textit{segunda inversión} y si se construye sobre la séptima, está en \textit{tercera inversión}.

	\end{itemize}

	\bigskip

	\item \textbf{Alteraciones}: signos que modifican la entonación o altura de los sonidos naturales y alterados. Las alteraciones más utilizadas son el \textit{sostenido} \sharp{} , el \textit{bemol} \flat{} y el \textit{becuadro} \natural{}. 

	\begin{itemize}

		\item \textbf{Alteraciones propias}: aquellas que se colocan al principio del pentagrama. Éstas son determinadas por la \textit{tonalidad} de la obra. También se les llama \textit{armadura}.

		\item \textbf{Alteraciones accidentales}: aquellas que aparecen en cualquier punto de la partitura, modificando temporalmente a la nota que acompañan.

	\end{itemize}

	\bigskip

	\item \textbf{Análisis armónico}: estudio, a partir de la tonalidad de la obra, de los \textit{acordes}, \textit{grados}, \textit{inversiones} y \textit{funciones} que componen a la misma y las relaciones entre ellos.

	\bigskip

	\item \textbf{Análisis melódico}: estudio de la melodía o línea melódica para determinar si está bien construida, es decir, si es fluida y consistente con la \textit{armonía} sobre la que se basa.

	\bigskip

	\item \textbf{Armadura}: conjunto de \textit{alteraciones} propias de una tonalidad escritas al principio de un pentagrama, entre la \textit{clave} y el \textit{compás}. Determina qué notas deben ser interpretadas de manera sistemática un semitono por encima o por debajo de sus notas naturales (no alteraradas) equivalentes.

	\bigskip

	\item \textbf{Armonía}: disciplina de la música en la que se estudian los acordes y sus relaciones para producir sensaciones sonoras, sensaciones de relajación o sosiego (consonancia)  y de tensión (disonancia). 

	\begin{itemize}

		\item \textbf{Armonía clásica} o \textbf{armonía tradicional}: referida a la armonía tonal o funcional. La armonía tonal es la empleada desde la época Prebarroca (siglo XV) hasta el Romanticisimo (siglo XIX). 

	\end{itemize}

	\bigskip

	\item \textbf{Coherencia musical}: decimos que una obra o melodía es coherente cuando ésta tiene fluidez y es consistente.

	\bigskip

	\item \textbf{Conducción melódica}: referida a la dirección y \textit{movimientos} entre las notas de una melodía. 

	\bigskip

	\item \textbf{Consonancia}: percepción subjetiva según la cual se consideran ciertos \textit{intervalos} musicales menos tensos que otros. Conjunto de sonidos que el oído percibe de forma distendida, agradable.

	\bigskip

	\item \textbf{Contrapunto}: técnica de composición musical que evalúa la relación existente entre dos o más voces independientes  con la finalidad de obtener cierto equilibrio armónico.

	\bigskip

	\item \textbf{Compás}: entidad métrica musical compuesta por varias unidades de tiempo (\textit{figuras musicales}) que se organizan en grupos. Se especifica cuántos \textit{pulsos} o partes hay en cada compás y qué figura musical define un pulso.

	\bigskip

	\item \textbf{Coral}: obra para canto al unísono (una sola melodía) o polifónico (varias melodías).

	\bigskip

	\item \textbf{Disonancia}: percepción subjetiva según la cual se consideran ciertos \textit{intervalos} musicales más tensos que otros. Conjunto de sonidos que el oído percibe de forma desagradable o tiende a rechazar.

	\bigskip

	\item \textbf{Escala}: conjunto de sonidos ordenados, notas de un entorno sonoro particular, de manera simple y esquemática. Estos sonidos puede estar dispuestos de forma ascendente (de grave a agudo) o de forma descendente (de agudo a grave), uno a uno en posiciones específicas dentro de la escala, llamadas \textit{grados}.

	\bigskip

	\item \textbf{Estructura armónica}: referido al conjunto de \textit{acordes} que conforman una obra. 

	\bigskip

	\item \textbf{Figura musical}: signo que representa gráficamente la duración de una nota. Las figuras más conocidas son la redonda \wholeNote{}, blanca \halfNote{}, negra \quarterNote{}, corchea \eighthNote{}, semicorchea \sixteenthNote{} y fusa \thirtysecondNote{}.

	\bigskip

	\item \textbf{Forma musical}: referida a la estructura de la obra o tipo. Ejemplos: coral, sonata, sinfonía, canción, ...

	\bigskip

	\item \textbf{Función tonal} o \textbf{función armónica}: referida al grado de reposo o de inestabilidad (tensión) que aportan los \textit{grados} de una tonalidad. Las funciones tonales son \textit{tónica} (pasajes en reposo), \textit{subdominante} (pasajes intermedios entre reposo e inestabilidad)  y \textit{dominante} (pasajes inestables). 

	\bigskip

	\item \textbf{Grado tonal}: referido a la posición de cada nota dentro de una \textit{escala}. Se designan mediante números romanos correlativos I, II, III, IV, V, VI y VII. Cada grado tiene asociada una \textit{función tonal}:

	\begin{itemize}

		\item[--] I: función de \textit{tónica}.
		\item[--] II: función de \textit{subdominante}.
		\item[--] III: función de \textit{tónica}.
		\item[--] IV: función de \textit{subdominante}.
		\item[--] V: función de \textit{dominante}.
		\item[--] VI: función de \textit{tónica}.
		\item[--] VII: función de \textit{dominante}.

	\end{itemize}

	\bigskip

	\item \textbf{Intervalo}: diferencia de altura entre dos notas, medida cuantitativamente (número) en \textit{grados} o notas naturales y cualitativamente (especie) en tonos y semitonos. Cuantitativamente pueden ser segundas, terceras, cuartas, quintas, sextas, séptimas u octavas. Cualitativamente puede ser mayores, menores, justos, aumentados o disminuidos.

	\bigskip

	\item \textbf{Movimientos}: referido a los movimientos que pueden darse entre dos notas. Los movimientos pueden ser a nivel melódico o armónico.

	\begin{itemize}

		\item \textbf{Movimientos melódicos}: movimientos que se produce al pasar de una nota a otra en una melodía. Puede ser un movimiento por grado conjunto - nos movemos a la nota continua ascendente o descendente - o un movimiento por salto - nos movemos a una nota no continua.

		\item \textbf{Movimientos armónicos}: movimientos que se producen en las notas -dos a dos- al pasar de un acorde a otro. Pueden ser movimientos \textit{directos} (ambas notas se mueven en la misma dirección) , movimientos \textit{paralelos} ( ambas notas se mueven en la misma dirección y manteniendo el mismo intervalo entre ellas) , movimientos \textit{contrarios} (ambas notas se mueven en direcciones contrarias) y movimientos \textit{oblicuos} (una nota se mantiene estática y la otra se mueve ascendente o descendentemente).

	\end{itemize}

	\item \textbf{Pulso}: unidad básica que se emplea para medir el tiempo.

	\bigskip

	\item \textbf{Sensible}: referida a la séptima nota de la escala que se encuentra a un semitono de la tónica de la tonalidad. También puede hacer referencia al VII grado de la escala, al acorde que se forma sobre dicho grado o a la función tonal que desempeña (dominante). 

	\bigskip

	\item \textbf{Séptima de dominante}: referida a la séptima del acorde que se forma sobre el acorde de \textit{dominante} de la \textit{tónica}.

	\bigskip

	\item \textbf{Tonalidad}: conjunto de sonidos que están en íntima relación entre sí con respecto a un centro tonal o \textit{tónica}.

	\bigskip

	\item \textbf{Tónica}: primer \textit{grado} de la \textit{escala} musical y es la nota que define la \textit{tonalidad}. 

\end{itemize}