\thispagestyle{empty}

\begin{center}
{\large\bfseries Desarrollo de un Sistema Experto de Análisis Musical}\\
\end{center}
\begin{center}
Laura Olga Tirado López\\
\end{center}

%\vspace{0.7cm}
\noindent{\textbf{Palabras clave}: sistema, experto, análisis, analizar, música, musical, coro, coral, quintas, octavas, armonía, armónico, melodía, melódico, contrapunto, voces, partitura}\\

\vspace{0.7cm}
\noindent{\textbf{Resumen}}\\

Se pretende desarrollar una aplicación basada en reglas para poder analizar \textit{corales} o partituras de coro.

\bigskip

El objetivo es implementar un sistema experto que permita a los usuarios analizar partituras de coro a cuatro voces para su posterior corrección, pudiendo así mejorarlas tanto a nivel armónico como a nivel melódico y de contrapunto.

\bigskip

El sistema se basará en conocimiento experto sobre lenguaje musical, composición y armonía clásica proporcionado por varios profesores de armonía y composición del Conservatorio Profesional Ángel Barrios, músicos titulados y otras fuentes tales como libros y tratados. 

\bigskip

Las funcionalidades que proporcionará este sistema se dividirán en dos clases:

\begin{itemize}
	\item \textbf{Análisis armónico}: consistente en la búsqueda de errores que comprendan, al menos dos de las voces, tales como acordes incompletos, secuencias armónicas incorrectas y búsqueda de disonancias.
	\item  \textbf{Análisis melódico}: consistente en la búsqueda de errores referidos a las melodías de las voces, tales como conducción incorrecta de las mismas, falta de coherencia y saltos o movimientos disonantes. 
\end{itemize}

\bigskip

El interés principal de este proyecto es poder desarrollar una herramienta para compositores o estudiantes de composición que haga una función análoga a la de un corrector ortográfico y gramatical de un editor de textos, aplicada a la composición y edición de partituras.

\cleardoublepage


\thispagestyle{empty}


\begin{center}
{\large\bfseries Development of an Expert System for Musical Analysis}\\
\end{center}
\begin{center}
Laura Olga Tirado López\\
\end{center}

%\vspace{0.7cm}
\noindent{\textbf{Keywords}: system, expert, analysis, analize, music, musical, choir, fifths, eights, harmony, harmonic, melody, melodic, voices, sheet, counterpoint}\\

\vspace{0.7cm}
\noindent{\textbf{Abstract}}\\

My intention is develop an app that implements an expert system to analysis \textit{choir} sheets.

\bigskip

This development aims to create a rule bases engine that allows users analize choir sheet musics for later correction, being able to improve their harmonies, melodies and counterpoint.

\bigskip

The system will be based on expert knwoledge about music notation, composing and classic harmony provided by many harmony and composition teachers of Conservatorio Profesional Ángel Barrios, musicians and another sources such as books and musical tratrises.

\bigskip

Features provided by this system will be of two grades:

\begin{itemize}
	\item \textbf{Harmonic analysis}: consisting on search for errors involving, at least, two of the voices, such as incomplete chords, incorrect harmonic sequences and  dissonances.
	\item  \textbf{Melodic analysis}: consisting on search for errors involving melodies, such as incorrect voice guidance, lack of coherence and dissonant movements. 
\end{itemize}

\bigskip

The main interest of this project is being able of develop a tool for composers or music students that performs a function analogous to an spelling and grammar checker, applied to composing and writing music. 

\newpage
\thispagestyle{empty}

\noindent\rule[-1ex]{\textwidth}{2pt}\\[4.5ex]

D. \textbf{Juan Luis Castro Peña}, Profesor del Área de XXXX del Departamento de Ciencias de la Computación e Inteligencia Artificial de la Universidad de Granada.

\vspace{0.5cm}

\textbf{Informa:}

\vspace{0.5cm}

Que el presente trabajo, titulado \textit{\textbf{Desarrollo de un Sistema Experto de Análisis Musical}},
ha sido realizado bajo su supervisión por \textbf{Laura Olga Tirado López}, y autorizamos la defensa de dicho trabajo ante el tribunal que corresponda.

\vspace{0.5cm}

Y para que conste, expiden y firman el presente informe en Granada a X de mes de 2017.

\vspace{1cm}

\textbf{El director:}

\vspace{5cm}

\noindent \textbf{Juan Luis Castro Peña}

\chapter*{Agradecimientos}
\thispagestyle{empty}

       \vspace{1cm}

\bigskip
A mi familia, por apoyarme y hacer posible que esté donde estoy hoy.

\bigskip
A mis amigos, por estar siempre ahí y animarme a seguir cuando las cosas se tuercen.

\bigskip
A mis compañeros de informática, porque sin ellos esta etapa que es la vida universitaria no habría tenido tantas risas.

\bigskip
A mis compañeros de la DEIIT, siempre dispuestos a escuchar mis quejas y ayudarme con mis problemas.